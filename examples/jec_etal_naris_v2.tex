% UPDATES:
% - figures, captions
% - methods: phase and kCSD
% - paragraph about the phase and kCSD in the results
% - bibliography cleaned up and added kCSD citation. 
%  EC July 2 AM

\documentclass[11pt]{article}

% packages
\usepackage{natbib}
\usepackage{graphicx}
\usepackage[nolists]{endfloat}
\usepackage{times}
\usepackage{ifthen}
\usepackage{parskip}
\usepackage[font=sf,labelfont=bf]{caption}
\usepackage{xspace}
\usepackage[pdftex]{color}
\usepackage{pdfcolmk}
\usepackage{fixltx2e} % for \textsubscript


% line numbers
\usepackage[left]{lineno}

% variable margins
\usepackage[left=2.5cm,top=2.5cm,bottom=3.5cm,right=2.5cm]{geometry}

% this helps figure placement
\renewcommand{\textfraction}{0.0}
\renewcommand{\topfraction}{1}
\renewcommand{\bottomfraction}{1}

% spacing
\setlength{\parindent}{0in} 
\setlength{\parskip}{2\baselineskip}
\linespread{2}
\renewcommand{\baselinestretch}{1.66}\normalsize

% definitions
\newcommand{\bsf}[1]{\textbf{#1}}
\newcommand{\sem}{S.E.M.\@\xspace}
\newcommand{\degree}{$^o$\@\xspace}
\makeatletter
\setlength{\@fptop}{0pt}
\makeatother

% bib
\bibliographystyle{apa}
\let\cite=\citep
\let\citeN=\citet
\let\citeNP=\citealt
\renewcommand{\bibfont}{\footnotesize}
\setlength{\bibsep}{2pt}

\begin{document}

{\Large\bf Gamma oscillations in the rat ventral striatum originate in the piriform cortex}

{\bf Authors}: James E.\ Carmichael\textsuperscript{1}, Jimmie
M.\ Gmaz\textsuperscript{1}, Matthijs A.\ A.\ van der
Meer\textsuperscript{1*}

\textsuperscript{1}Department of Psychological and Brain Sciences,
Dartmouth College, Hanover NH
03755\\ %\textsuperscript{2}Department of Biology and
%\textsuperscript{3}Centre for Theoretical Neuroscience, University of
%Waterloo, Canada\\

\textsuperscript{*}Correspondence should be addressed to MvdM,
Department of Psychological and Brain Sciences, Dartmouth College, 3
Maynard St, Hanover, NH 03755. E-mail: {\sffamily mvdm@dartmouth.edu}.

\textbf{Number of Figures:} 8\\
\textbf{Number of Tables:} 0\\
\textbf{Total Word Count:} 5631\\
\textbf{Abstract Word Count:} 174\\
\textbf{Introduction Word Count:} 549\\
\textbf{Discussion Word Count:} 1474\\

\textbf{Acknowledgments}: We thank Nancy Gibson, Martin Ryan and Jean
Flanagan for animal care, Claire Cheetham for suggesting the naris
occlusion technique, Youki Tanaka for reagents, and Min-Ching Kuo and
Alyssa Carey for technical assistance. This work was supported by
Dartmouth College (Dartmouth Fellowship to JEC, and start-up funds to
MvdM) and the Natural Sciences and Engineering Research Council
(NSERC) of Canada (Discovery Grant award to MvdM, Canada Graduate
Scholarship to JMG).

\textbf{Conflict of Interest}: The authors declare no competing
financial interests.\\

\newpage
\linenumbers

\section*{Abstract}

Local field potentials (LFP) recorded from the human and rodent
ventral striatum (vStr) exhibit prominent, behaviorally relevant
gamma-band oscillations. These oscillations are related to local
spiking activity and transiently synchronize with anatomically related
areas, suggesting a possible role in organizing vStr
activity. However, the origin of vStr gamma is unknown. We recorded
vStr gamma oscillations across a 1.4mm\textsuperscript{2} grid spanned
by 64 recording electrodes as rats rested and foraged for rewards,
revealing a highly consistent power gradient originating in the
adjacent piriform cortex. Phase differences across the vStr were
consistently small ($<$10\degree) and current source density analysis
further confirmed the absence of local sink-source pairs in the
vStr. Reversible occlusions of the ipsilateral (but not contralateral)
nostril, known to abolish gamma oscillations in the piriform cortex,
strongly reduced vStr gamma power and the occurrence of transient
gamma-band events. These results imply that local circuitry is not a
major contributor to gamma oscillations in the vStr LFP, and that
piriform cortex is an important driver of gamma-band oscillations in
the vStr and associated limbic areas.

\section*{Significance Statement (120 words)}

The ventral striatum is an area of anatomical convergence in circuits
underlying motivated behavior, but it remains unclear how its inputs
from different sources interact. A major proposal of how neural
circuits may dynamically switch between convergent inputs is through
temporal organization reflected in local field potential (LFP)
oscillations. Our results show that in the rat, the mechanisms
controlling vStr gamma oscillations are primarily located in the in
the adjacent piriform cortex, rather than vStr itself. This provides a
novel interpretation of previous rodent work on gamma oscillations in
the vStr and related circuits, and an important consideration for
future work seeking to use oscillations in these areas as biomarkers
in rodent models of human behavioral and neurological disorders.

\newpage

\section*{Introduction}
% Version 2.1 - edited MvdM 2015-08-04
% MvdM edit 2016-12-01 (!)

The ventral striatum (vStr) is an anatomically heterogeneous region
receiving a convergence of anatomical connections from structures in
the cortico-striatal-thalamic loop, as well as from the hippocampal
formation and amygdala
\cite{Pennartz94,Haber2009,Sesack2010}. Prominent local field
potential (LFP) oscillations can be recorded from the rodent and human
vStr, spanning a broad range of frequencies that include delta, theta,
beta, multiple gamma bands, and high-frequency oscillations
\cite{Berke2004a,VanderMeer2010b,Axmacher2010,Durschmid2013,Hunt2009a}
and display phase-amplitude coupling (humans: \citeNP{Cohen2009f},
rodents: \citeNP{Donnelly2014,Malhotra2015a}). A particularly salient
feature of the vStr LFP is the presence of prominent gamma-band
oscillations, which include distinct low-gamma ($\sim$45-65 Hz) and
high-gamma ($\sim$70-90 Hz) components.

Several studies have found correlations between the occurrence of
these low- and high-gamma oscillations such as reward approach and/or
receipt (\citeNP{Vandermeer2009a,Berke2009b}, but see
\citeNP{Malhotra2015a}), reward expectation and prediction errors
\cite{Cohen2009f,Axmacher2010}, drug-related conditioned place
preference \cite{Dejean2016}, and impulsive actions
\cite{Donnelly2014}. vStr gamma oscillations are also affected by
manipulations of the dopamine and cannabinoid systems
\cite{Berke2009b,Lemaire2012, Morra2012}. Importantly, the spiking of
vStr neurons is related to these gamma oscillations, with putative
fast-spiking interneurons (FSIs) displaying a particularly striking
pattern: partially distinct populations appear to phase-lock to low-
and high-gamma respectively \cite{Berke2005,Vandermeer2009a,
  Dejean2016}. Medium spiny neurons (MSNs), which make up the vast
majority of vStr neurons, tend to show weaker but nonetheless
statistically significant phase locking \cite{Kalenscher2010,Howe2011}
and ensemble spiking activity can predict which gamma band oscillation
is present in the LFP \cite{Catanese2016}.

 A conservative interpretation of the above body of work is to view
 gamma oscillations in the vStr LFP as a signal containing a certain
 amount of information about the state of the local network, that is,
 as a practically useful readout. However, in several brain structures
 there is evidence that the effectiveness of an incoming stimulus can
 depend on the phase of local gamma oscillations
 \cite{Cardin2009,Sohal2009}; although such state-dependence has not
 yet been shown in the vStr, the findings reviewed so far suggest the
 possibility that the mechanisms generating vStr gamma contribute to
 dynamic gain control, as in ``communication through coherence''
 proposals \cite{Akam2010,Womelsdorf2014,Fries2015}. This idea is
 conceptually attractive given the anatomical convergence of multiple
 inputs onto the vStr, which all oscillate
 \cite{Odonnell1995,Gruber2009,Harris2015}; indeed, gamma oscillations
 in the vStr LFP are often coherent with gamma-band LFP signals in
 anatomically related areas, with prefrontal cortical areas being the
 best-studied example (rodent:
 \citeNP{Berke2009b,McCracken2009,Dejean2013,Catanese2016}, human:
 \citeNP{Cohen2009d}).

For both the ``readout'' or ``gain control mechanism'' views of vStr
gamma oscillations, it is important to establish how this signal is
generated; a crucial first step is to localize their source(s), which
have so far remained unclear.  Some studies emphasize the similarity
of vStr gamma oscillations with those recorded in the adjacent
piriform cortex \cite{Berke2009b} while other studies report local
heterogeneities that appear to be inconsistent with volume conduction
\cite{Kalenscher2010,Morra2012}, or focus on cell-intrinsic
contributions such as gamma-band resonance
\cite{Taverna2007}. Resolving this issue would help determine if local
vStr circuitry may implement a ``switchboard'' through a
communication-through-coherence type mechanism
\cite{Fries2005a,Gruber2009,Akam2010}, and to enable a productive
dialog between rodent and human studies in which vStr LFPs have
behavioral and clinical relevance
\cite{Sturm2003,McCracken2009,Dejean2013}.

Thus, to determine the origin of gamma oscillations in the local field
potential of the rat vStr, we (1) record from across the vStr using a
high density electrode array, and (2) inactivate the piriform cortex
using reversible naris (nostril) occlusions known to abolish piriform
gamma \cite{Zibrowski1997}.


\section*{Methods}
%Version 3.1 - minor edits MvdM 2015-07-13
%Version 3.2 - timeline updates Eric 2015-07-14_am
%Version 4.0 - phase and kCSD EC 2016-07-02_am
%Version 4.1 - sessions used per rat; CSD update

{\bf Overview.} This study consists of two experiments: (1) a ``LFP
mapping'' experiment measuring the distribution of gamma oscillations
across the vStr, and (2) a ``naris occlusion'' experiment testing the
effects of unilateral nostril closures on vStr gamma. Data in the LFP
mapping experiment were acquired as rats performed a maze-based
foraging task, as well as during off-task resting periods. Since we
found no difference in the properties of gamma oscillations on- and
off-task, data for the naris experiment were acquired during rest
only. All procedures were approved by the University of Waterloo
Animal Care Committee (AUPP 11-06) and the Dartmouth College IACUC (vand.ma.2).

{\bf Subjects and timeline}. Seven Long-Evans male rats
(\textgreater10 weeks old; \textgreater400g) were used in total (four
in the LFP mapping experiment and three in the naris occlusion
experiment).  For the LFP mapping experiment, rats were pretrained on
a foraging task (4 days of maze habituation), implanted with recording
probes (described in detail below), retrained on the task following
recovery (minimum 4 days), and recording data acquired.  The naris
experiment included two na\"{\i}ve rats and a final rat that had been
previously trained on set-shifting task used in Gmaz et al. (SfN
abstract, 2016; no differences were found across rats). All animals
were kept on a 12hr light/dark cycle with all the experiments
performed during the light phase.

{\bf Surgery and recording probes}. Rats for the LFP mapping
experiment were implanted with 64 channel silicon probes
(A8x8-10mm-200-200-177, NeuroNexus; 50$\mu$m thick). Probe recording
sites and tetrodes were gold-plated (Sifco 6355) to impedances between
300-500k$\Omega$ (Nano-Z, White Matter LLC). Probes were attached to a
microdrive and implanted as in \citeN{Vandecasteele2012}, with the
addition of a separate independently movable stainless steel wire
reference electrode (gold-plated to 50k$\Omega$) implanted into the
same hemisphere (AP 2.2 mm anterior to bregma, ML 2.0 mm, targeting
the corpus callosum overlying the vStr).  The probes contained
regularly spaced recording electrodes, spanning
1.4mm\textsuperscript{2} and arranged in a 8x8 grid, and were
implanted rotated around the vertical axis with the edges between AP
0.6--2.28, ML 1.4--2.8) (Figure \ref{fig:trace}C).  Rats for the naris
occlusion experiment were implanted with a either a custom built
4-tetrode multi-site drive, with one tetrode located in the vStr (AP
1.5, ML 2.0), with additional tetrodes in the orbitofrontal,
prelimbic, and cingulate cortices (not analyzed here), or a custom
16-tetrode ``hyperdrive'' with all tetrodes in the vStr.  Following
surgery, probes and tetrodes were moved down over the course of
several days until they reached the target region (DV 6.5--8 mm).

{\bf Behavioral task.} The {\it LFP mapping} rats were trained on a
square-shaped elevated track (width 10 cm, each edge of the square
100cm long) with sucrose reward receptacles placed at each of the four
corners. Nosepoking in the reward receptacles yielded 0.1, 0, 0.1, or
0.2ml of 12\% sucrose reward respectively; with experience (daily
40-minute sessions, preceded and followed by 5-10 minutes of off-task
recording on a terracotta planter filled with towels) animals learn to
skip the non-rewarded site. Throughout behavioral training and
recording on the task, rats were food-restricted to \textgreater85\%
of their free-feeding weight to encourage foraging behavior. The {\it
  naris occlusion} rats had {\it ad libitum} access to food. Before
the start of the experiments described here, one of the three animals
had been previously trained on a behavioral task that used the same
physical setup in the same room (Gmaz et al.\ SfN abstract, 2016).

{\bf Data acquisition and preprocessing.}  Wideband signals were
acquired for the silicon probe and ``hyperdrive'' recordings using
RHA2132 v0810 multiplexing headstages (Intan) and a KJE-1001/KJD-1000
amplification system (Amplipex), sampled at 20kHz (decimated to 2kHz
during analysis) referenced against a single electrode in the corpus
callosum above the vStr.  To extract spikes from the continuously
sampled Amplipex data, the data was filtered (600-9000Hz), thresholded
and peak-aligned (UltraMegaSort2k, \citeNP{Hill2011}).  Recordings for
the na\"{\i}ve multisite 4-tetrode drive rats used a Digital Lynx data
acquisition system with an HS-18mm preamplifier (Neuralynx) with the
low pass 0.5-500 Hz subsampled from 32kHz to 2KHz.  Spiking data the
voltage was recorded at 32kHz when the voltage exceeded a predefined
threshold.  Data for LFP mapping analysis had the DC offset removed
and was filtered (1-500Hz, 10th order Butterworth, {\tt
  filtfilt.m}). Putative single neurons were manually sorted using
MClust 3.5 (A.D.\ Redish et al.).  Electrode sites with irregular
impedance values (\textgreater900k$\Omega$), intermittent signals, or
known defective sites (B-stock probes) were excluded from analysis
(black pixels in Figures \ref{fig:distrib}-\ref{fig:phase}).

{\bf Naris occlusion}. Reversible naris closures (``nose plugs'') were
constructed from PE90/100 tubing (Intramedic) by tying a human hair
and a suture in a double knot, threading it through the side of the
tubing, and gluing it to the inside of the tube such that it protruded
$\sim$8mm beyond the tubing, facilitating subsequent removal
\citep{Kucharski1987, Cummings1997}. Nose plugs were coated with
Vaseline and inserted (or removed) while the rat was briefly under
isoflurane anesthesia ($\le$2 min from the time of induction). The
effectiveness of this procedure in blocking air flow through the
occluded nostril was verified by visual inspection: any remaining
airflow tended to move the Vaseline. Daily recording sessions
consisted of four off-task segments: a non-occlusion baseline
(``pre''), ipsilateral and contralateral naris occlusions (order
counterbalanced across sessions) and another non-occlusion baseline
(``post'') separated by 45 minutes to minimize any effects of the
isoflurane anesthesia (Figure \ref{fig:naris}G).

{\bf Session inclusion criteria}. For each {\it LFP mapping} subject,
data from three consecutive daily recording sessions were analyzed,
with the recording probe at the same depth across days. These sessions
were chosen such that the probe had to be at the same depth across
sessions, and that either the animal had reached the performance
criterion on the task, and if they did not (2/4 rats) then the final
three sessions at a consistent depth were used. For one LFP mapping
subject, only two sessions were included due to a faulty recording
tether. For the naris occlusion experiment, four days of data were
recorded and analyzed (two days, followed by one day of no recording,
and two more days).

{\bf Data analysis overview}.  All analyses were performed using
MATLAB 2014a and can be reproduced using code available on our public
GitHub repository (http://github.com/vandermeerlab/papers) with data
files, metadata, and code usage guides available upon request. We
performed spectral analysis on the LFP data using power spectral
densities (PSDs) over specific epochs within recording sessions
(off-task baseline recordings before and after behavior; reward site
approach (run) and reward site consumption; and rest data only for the
naris occlusion group). Given that gamma oscillations in the vStr LFP
tend to occur in distinct events
\cite{masimore05,Cohen2009d,Donnelly2014} we also performed
event-based analysis (detailed below).

% updated to reflect newer gamma detection method
{\bf Gamma event detection and analysis}. Gamma events were detected
following the procedure in \citeNP{Catanese2016}. First, LFP data were
filtered into the bands of interest (low-gamma: 45-65 Hz; high-gamma:
70-90 Hz) using a 5\textsuperscript{th} order Chebyshev filter (ripple
dB 0.5) with a zero-phase digital filter (MATLAB {\tt
  filtfilt()}). The instantaneous amplitude was extracted from the
filtered signal by Hilbert transform (MATLAB {\tt
  abs(hilbert())}). For the LFP mapping experiment, a single recording
channel for gamma event detection was selected by first computing the
power spectral density (PSD) for the six most ventrolateral recording
sites and using the site with the largest power averaged across
frequencies in the low-gamma range (45-65Hz). Large amplitude
artifacts ($>$4 SDs from the mean in the unfiltered data) and chewing
artifacts ($>$3SD of the z-scored data filtered in the 200-500Hz) were
first removed from the data. If the instantaneous amplitude surpassed
the 95\textsuperscript{th} percentile of the amplitude of the session
in it was considered a candidate event. Events were excluded if they:
co-occurred with high voltage spindles ($>$4SD in 7-11Hz), had $<$4
cycles, or had a variance score (variance$/$mean of cycle peaks and
troughs) $>$1.5.  Applying a threshold (40$\mu V$) to the minimum
filtered amplitude helped remove false positives in the high-gamma
band only, but did not aid in rejecting false positives for low-gamma.
% yes these are all used in the final versions of the gamma detection
script for both the mapping and naris data.

For each detected gamma event, a length-matched pseudo-random
``control'' event was extracted by identifying the period of lowest
amplitude in the gamma band of interest within a 20s window prior to
each detected gamma event. Gamma and random events were then converted
into one of two formats.  For gamma power analysis, events were
converted to the FieldTrip format by taking 100ms of data centered on
detected gamma events ({\tt ft\_redefinetrial}, FieldTrip toolbox,
\citeNP{Oostenveld2011}). For phase and current source density (CSD)
analysis the events were converted into a three cycle "triplet" by
identifying the cycle with the highest amplitude and extracting it as
well as the cycle before and after.  The three extracted cycles were
then interpolated to ensure that each event had an equal number of
samples (214) allowing for averaging across events.

For the naris occlusion experiment events were detected using the
percentile method detailed above, except that a threshold value (in
microvolts) was obtained from the baseline data and then applied to
the ipsi- and contralateral sessions; this was necessary to be able to
compare the number of gamma events between conditions.

{\bf Spectral analysis}. Session-wide PSDs were computed for the naris
experiment using Welch's method ($\sim$2s Hanning window) on the first
derivative of the data to remove the overall 1/f trend (MATLAB {\tt
  pwelch(diff(data))}).  Event-based gamma power was computed using
the FieldTrip toolbox \citep{Oostenveld2011} function {\tt
  ft\_freqanalysis} ('mtmfft', Hanning window, frequency of interest
(FOI) 1-500Hz, window 5./FOI) by averaging the frequencies of interest
in the resulting PSD. Phase differences were computed relative to the
ventrolateral most electrode using cross-power spectral density
(MATLAB {\tt cpsd}) using a 64 sample Hamming window with 50\% overlap
(NFFT = 1024).

% {\color{red} not sure how to bring up the different   types of events FT/regular.. we discussed this right? Either 100ms ft, or 3 center cycles as I recall -- MvdM}

{\bf Plane fits}. To quantify the consistency of any patterns in the
gamma power distribution across probe recording sites we computed a
plane of best fit using least squares for the gamma power across the
array during each gamma event and compared the variance explained
(R\textsuperscript{2}) to the pseudo-random non-gamma events of
equivalent length.

{\bf Current source density analysis}. Current source densities (CSD)
were computed by taking the second spatial derivative of the bandpass
filtered data across the recording channels along the dorsomedial to
ventrolateral diagonal of the recording array and multiplying it by
the conductance (0.3mS/mm).  Missing channels along this diagonal were
filled in by interpolation (MATLAB {\tt griddata}). Average CSDs for
each rat were computed using the three cycle triplet from each
detected gamma event (mentioned above). %how to cite?

\section*{Results}

%\subsection*{Experiment 1: Silicon recording arrays reveal a strong
%  linear gradient in gamma band oscillations across the ventral striatum}
\subsection*{Experiment 1: mapping of gamma oscillations in the
  ventral striatum} 

Clear gamma events were recorded across the vStr (Figure
\ref{fig:trace}A) of all rats chronically implanted (n = 4) with
64-channel planar silicon probes, which covered an area of
1.4mm\textsuperscript{2} with a regular 8x8 grid of recording sites
(Figure \ref{fig:trace}B and C). We recorded wideband neural data
during behavior on an elevated maze and during off-task resting
periods. As in previous reports, we found clear low- and high-gamma
oscillations in the LFP (Figure \ref{fig:trace}A;
\citeNP{Leung1993,Berke2004a,Howe2011}) using probes implanted in the
vStr, to which putative interneurons showed significant phase locking
(Figure \ref{fig:trace}D). Gamma oscillations appeared highly coherent
across sites, with visual inspection suggesting systematic changes in
power across sites. To quantify this effect, we isolated gamma events
using a thresholding procedure on the channel with the largest average
gamma-band power ({\it Methods}); examples of detected events are
shown as shaded areas in Figure \ref{fig:trace}A (low-gamma: blue,
high-gamma, green). We plotted the gamma-band power across all probe
sites as a heat map, illustrated for an example low-gamma event in
Figure \ref{fig:distrib}A and B. An approximately linear gradient was
apparent, with the highest power at the ventrolateral pole and the
lowest power at the dorsomedial pole (Figure \ref{fig:distrib}B).

\begin{figure}[h]
\centering
\includegraphics[width=\textwidth]{Naris_Figure_0_Feb}
\caption{High-density planar silicon probe recordings across the
  ventral striatum.  \bsf{A}: Example ventral striatum LFP traces
  (filtered between 1-500 Hz; the two sets are from different subjects
  R1 and R2) ordered by the location of the corresponding electrode
  site along the ventrolateral (VL, top) to dorsomedial (DM, bottom)
  axis of the probe. Gamma events were detected using a thresholding
  procedure; example low-gamma events are shaded in blue, high-gamma
  events in green. Note that the amplitude of the gamma events
  increases from the dorsomedial to ventrolateral electrodes. \bsf{B}:
  Recording locations as determined by histological processing. For
  silicon probes (inset), recording locations are indicated by the
  corners of the probe (o+x\#), while tetrode locations are marked
  with dots (colored by subject) \bsf{C}: Dorsal view of the electrode
  locations to demonstrate the angle of the probes (lines), and
  location of tetrodes (dots) all colored by subject.  \bsf{D}: Field
  pairwise phase consistency (PPC, \protect\citeNP{Vinck2011}) of a
  representative ventral striatal unit showing a clear peak in the
  gamma band.}
\label{fig:trace}
\end{figure}

\begin{figure}[h]
\centering
\includegraphics[width=\textwidth]{Naris_Figure_1_Feb}
\caption{Gamma band local field potential (LFP) oscillations in the
  ventral striatum form a dorsomedial to ventrolateral power gradient.
  \bsf{A}: Raw LFP traces (left) from the recording electrodes at the
  dorsomedial (o), ventromedial (x), dorsolateral (*) and
  ventrolateral (\#) points of the silicon probe for a representative
  low-gamma event.  Grey bar spans the length of the gamma event (see
  {\it Methods} for details of gamma detection). \bsf{B} Heat map
  (right) showing the gamma power ($\mu V^2$) across the recording
  array (64 sites, regularly spaced in an 8x8 grid spanning
  1.4mm\textsuperscript{2}) during the same low-gamma event as seen in
  the raw traces (left). Gamma power is about five times greater in
  the ventrolateral region compared to the dorsal-medial
  region. \bsf{C}: Average low- and high-gamma power across an entire
  recording session for each subject using the same probe layout in
  similar recording locations across animals (scale is normalized to
  the lowest power channel). Black spaces represent defective
  recording sites (see {\it Methods} for defective site criteria).}
\label{fig:distrib}
\end{figure}

As a first step towards establishing the generality of this linear
gamma power gradient, we detected low-gamma (45-65 Hz) and high-gamma
(70-90 Hz) events across multiple recording sessions and subjects (11
sessions across 4 subjects; total events detected, 5997 for low-gamma,
5005 for high-gamma; mean event rates 0.32/s for low-gamma, 0.32/s for
high-gamma; mean $\pm$ SD event length, 140.89 $\pm$ 59.52 ms for
low-gamma, 81.30 $\pm$ 42.69 ms for high-gamma). Although the
distribution of functional sites varied between probes, the linear
gamma power gradient was consistent across subjects (Figure
\ref{fig:distrib}C).  Next, we asked if the distribution of gamma
power across the vStr was affected by activity level (running on an
elevated track for sucrose reward vs.\ off-task rest) or by different
behaviors during the task (approaching reward sites vs.\ reward
receipt and consumption). The observed gamma power gradient was highly
consistent across all these conditions for both low- and high-gamma
(Figure \ref{fig:plane}A). This systematic power gradient was further
confirmed by plotting gamma power as a function of distance from the
ventrolateral pole of the probe confirmed the systematic power
gradient (Figure \ref{fig:distance}A and B).

\begin{figure}[h]
\centering
\includegraphics[width=\textwidth]{Naris_Figure_2_Feb}
\caption{The vStr dorsomedial to ventrolateral gamma power gradient is
  conserved across different behaviors and is significantly different
  from randomly selected epochs.  \bsf{A}: Average gamma power
  distributions across vStr during active foraging and reward
  consumption for two subjects that completed the foraging
  task. \bsf{B}: Histograms of variance explained
  (R\textsuperscript{2}) for plane fits to low-gamma (left) and
  high-gamma (right) and shuffled events of equivalent length
  (red). Gamma event R\textsuperscript{2} values form a tight cluster
  with the majority of the variance explained by the best fit plane,
  while random events fail to fit the plane. Insets are representative
  gamma and shuffled events.}
\label{fig:plane}
\end{figure}

To determine if this gradient was simply a consequence of probe
impedance values or other peculiarities of the recording setup, we
compared the power gradient obtained during detected gamma oscillation
events to the power gradient from a set of randomly chosen control
events. By fitting a plane to the distribution of power across the
probe, we could quantify how much of the variance in power across
probe sites was accounted for by a linear power gradient. If this
gradient was due to nonspecific properties of the signal, then this
pattern would be similar during true gamma oscillations and non-gamma
events. Contrary to this scenario, the ventrolateral gradient
disappeared for the random control events (Figure \ref{fig:plane}B,
low-gamma events mean R\textsuperscript{2} 52.74 $\pm$ 24.08,
low-gamma matched random epochs mean R\textsuperscript{2} 22.79 $\pm$
20.94; high-gamma events mean 59.15 $\pm$ 15.71, high-gamma matched
random epochs mean 18.51 $\pm$ 18.11; independent samples t-test:
t\textsubscript{(2690)} = 34.43, p $<$ 0.001 for low-gamma;
t\textsubscript{(2308)} = 57.62, p $<$ 0.001 for
high-gamma). Furthermore, as reported previously \cite{Berke2004a},
high-voltage spindles (HVS) displayed a power gradient in the opposite
direction, with largest power at the dorsomedial pole (Figure
\ref{fig:distance}C inset). Thus, the ventrolateral power gradient
observed during gamma oscillations does not result from nonspecific
probe or recording system properties.


\begin{figure}[h]
\centering \includegraphics[width=\textwidth]{Naris_Figure_4_Feb}
\caption{Gamma events form a consistent dorsomedial to ventrolateral
  power gradient that can be separated from high voltage spindles and
  random epochs. Random epochs were duration matched periods of low
  gamma power (see {\it Methods} for details). \bsf{A}: Gamma power
  relative to distance from the ventrolateral most site on the
  probe. Average gamma power from each recording session (points)
  separated by subject (colors) normalized to the maximum value for
  both low (\bsf{A}) and high (\bsf{B}). \bsf{C}: PCA on the power
  gradients across events reveals that both low-gamma and high-gamma
  are indistinguishable from each other, yet are clearly separate from
  both random epochs and high-voltage-spindles.}
\label{fig:distance}
\end{figure}

The ventrolateral gamma power gradient identified above suggests the
lack of a local source (in the vStr) for these oscillations. To
determine the source of vStr gamma, we first plotted phase differences
relative to the ventrolateral recording site. Although the specific
pattern of these phase differences varied across subjects, these
differences were consistently small ($<$10\degree, Figure
\ref{fig:phase}A); in particular, there was no evidence of phase
reversals, a tell-tale sign of systematically arranged sink/source
pairs. The patterns of phase differences were consistent between low-
and high-gamma within subjects. Next, we applied current source
density (CSD) analysis, which in accordance with the near-zero phase
gradients showed only very small sink/source pairs, either in single
examples (Figure \ref{fig:csd}A and B) or when averaged across all
events (Figure \ref{fig:csd}C). Thus no it appears that no obvious
source of either low- or high-gamma oscillations exists within the
vStr.

\begin{figure}[h]
\centering
\includegraphics[width=\textwidth]{Naris_Figure_3_Feb}
\caption{Average phase differences across the center three cycles in
  each event.  Phase lags were found to have a gradient along the
  ventrolateral to dorsomedial poles, but lacked directional
  consistency. Each plot shows the average phase difference (in
  degrees) relative to the ventrolateral most electrode (‘\#’). Phase
  differences were negligible showing a small lag from ventrolateral
  to dorsomedial in 2/4 subjects. Heat maps range between -10 to 10
  degrees ($\pm$ 0.6ms)}
\label{fig:phase}
\end{figure}


\begin{figure}[h]
\centering
\includegraphics[width=\textwidth]{Naris_Figure_5_Feb}
\caption{Current source density (CSD) analysis of gamma
  events. \bsf{A}: Filtered low-gamma event (same as shown in Figure
  \ref{fig:distrib}B). Each trace represents a recording site on the
  diagonal of the silicon probe (inset), note the change in power
  along the dorsomedial to ventrolateral axis but very similar
  phases. \bsf{B}: Sample CSD over the same low-gamma
  event. Pseudocolor scale represents fractional values relative to a
  180\degree phase inversion (source/sink pair). Only a weak
  source/sink appears on the ventrolateral pole across electrodes,
  corresponding to the slight phase shift in the example
  traces. \bsf{C}: Average CSD across the center three cycles (grey
  lines) by subject event. Note that no clear source/sink pair
  emerges, consistent with the lack of a phase reversal in the
  gamma-band LFP.}
\label{fig:csd}
\end{figure}

\subsection*{Experiment 2: Unilateral naris occlusion strongly reduces vStr gamma
  power}
  
The clear gradient in vStr gamma power is consistent with
\citeN{Berke2009b}'s proposal that the adjacent piriform cortex, known
to generate strong gamma oscillations, may be the main source of vStr
LFP gamma. Given that piriform gamma is known to be abolished by
occlusion of the ipsilateral nostril (naris, \citeNP{Zibrowski1997})
we tested the effects of ipsilateral naris occlusions on vStr gamma
power by inserting removable nose plugs alternately in one nostril,
and then the other ({\it Methods}). Contralateral occlusions,
performed alternately before or after the ipsilateral condition,
provided a control for nonspecific (e.g.\ behavioral) effects of naris
blockage. Ipsilateral naris occlusions effectively abolished low- and
high-gamma power relative to the contralateral condition, and relative
to unoccluded conditions before (``pre'') or after (``post'', Figure
\ref{fig:naris}A-C and H). Although power spectral densities of
individual recording sessions varied, likely due to behavioral
differences such as mobility on the pot, gamma suppression was highly
consistent across sessions and subjects (Figure \ref{fig:naris}A-C).


\begin{figure}[h]
\centering
\includegraphics[width=\textwidth]{Naris_Figure_6_Feb}
\smallskip
\caption{Ipsilateral, but not contralateral, naris occlusion reduces
  gamma power and event occurrence in the vStr. \bsf{A}: Spectrogram
  across all four experimental phases in a single session. Arrows
  emphasize the clear gamma band power that disappears during the
  ipsilateral phase. \bsf{B}: Naris experiment timeline. Ipsi- and
  contralateral occlusion order was counterbalanced across
  days. \bsf{C-E}: Normalized power spectral densities (PSDs) of
  representative sessions from each rat (R5,6, and 7
  respectively). Each session shows a clear reduction in power within
  the gamma bands for the ipsilateral occlusion condition only (red
  line). Note that although PSDs differed between sessions
  (e.g.\ high-voltage spindles, 7-11 Hz, in the ``post'' condition in
  (\bsf{D}), the reduction in gamma power was highly consistent. PSDs
  were computed on the first order derivative of the data to remove
  the 1/f distributions. \bsf{F}: Comparison of the average number of
  detected gamma events per condition normalized to the unoccluded
  condition. The ipsilateral condition yielded significantly fewer
  events for the same recording duration (see main
  text). Contralateral occlusion increased the number of high-gamma
  events. Errorbars represent SEM. \bsf{G-H}: Comparison of the
  average power in each session/subject (R4-7) within the
  low-gamma/high-gamma band. Ipsi- and contralateral conditions were
  normalized to the unoccluded condition (average between pre and
  post).}
\label{fig:naris}
\end{figure}

%{\color{red} EC: describe results in terms of this key
  %comparison first for both low and high gamma, it's the most
 % informative -- as you do when talking about the event based analysis
  %in the next para. Then talk about the comparison to the control
  %(non-occlusion). Also note that some of the below numbers don't make
  %sense.. same t-stat shows up twice with different p levels?}  To
quantify this effect, we compared gamma power extracted from the power
spectral density during ipsilateral and contralateral occlusions
respectively, after normalizing to gamma power during non-occluded
control conditions (``pre'' and ``post'' recording epochs). Only
ipsilateral occlusion significantly reduced gamma power in both low-
and high-gamma bands to a mean of 0.48 (SEM $\pm$ 0.05) and 0.62
($\pm$ 0.04) of the control condition.  Paired t-tests confirmed the
reduction was indeed significant for both low-gamma
(t\textsubscript{(11)} = -10.64, p \textless 0.001) and high-gamma
(t\textsubscript{(11)} = -5.10, p \textless 0.001) relative to the
power during the contralateral occlusion.  The contralateral occlusion
failed to reduce low-gamma power compared to the control (1.00 $\pm$
0.06 of the control condition, t\textsubscript{(11)} = 0.02, p =
0.99).  Contralateral occlusion did produce a marked decrease in the
high gamma power relative to the control (0.87 $\pm$ 0.05 of the
control condition, t\textsubscript{(11)} = -2.66, p = 0.02).
Ipsilateral occlusion provided a significantly greater reduction in
gamma power compared to contralateral occlusion for both low-
(t\textsubscript{(11)} = -11.62, p \textless 0.001) and high-gamma
(t\textsubscript{(11)} = -10.79, p \textless 0.001).

This strong reduction in gamma power could be due to fewer gamma
events occurring, and/or events having lower gamma power. Gamma event
detection applied to the occlusion conditions yielded a significantly
lower number of gamma events in the ipsilateral occlusion recording
(pre- and post session event count normalized to 1; low-gamma events,
0.04 ($\pm$ 0.21); high-gamma: 0.13 ($\pm$ 0.03) compared to the
contralateral occlusion (low-gamma: 1.09 $\pm$ 0.19, paired t-test
t\textsubscript{(11)} = -5.73, p \textless 0.001; high-gamma: 0.77
$\pm$ 0.17, t\textsubscript{(11)} = -4.10, p \textless 0.005, Figure
\ref{fig:naris}D). The number of events in the ipsilateral occlusion
was significantly lower than number of detected events in the
unoccluded condition for low-gamma (t\textsubscript{(11)} = -44.93,
p\textless 0.001) and high-gamma (t\textsubscript{(11)} = -29.59, p
\textless 0.001). The contralateral occlusion did not differ compared
to the control for low-gamma events (t\textsubscript{(11)} = 0.47, p =
0.69), or high-gamma events (t\textsubscript{(11)} = -1.41, p =
0.19). Further supporting the robustness of this result, the
ipsilateral condition gamma power and gamma event count was lower than
the contralateral condition gamma power in every individual session,
without exception. Thus, ipsilateral naris occlusion resulted in a
strong reduction in gamma power which resulted in a reduction in the
number of events detected.

\section*{Discussion}
%van der meer et al. 2010 mentions the possibility of gamma 50 coming form the piriform and gamma 80 being more realted to HC inputs.  This might explain why 80 seems to be more persistant in the ispsi condition.  But this is not a difference that is obvious in the PSDs, only in the occurance of events.  Could have to do with the ability to detect 50/80 better than the other in some cases. 

We have demonstrated that (1) the power of gamma oscillations in the
ventral striatal local field potential increases along a clear
dorsomedial-to-ventrolateral gradient, (2) the phases of gamma
oscillations across the vStr are highly consistent, with no evidence
of reversals indicating a local sink/source pair, and (3) gamma
oscillations were strongly reduced by occlusion of the unilateral, but
not contralateral, nostril. Together, these results strongly suggest
that gamma oscillations in the vStr LFP are volume-conducted from
piriform cortex, consistent with initial observations by
\citeN{Berke2009b}, who reported highly similar LFPs in vStr and
piriform. Here, we build on this initial work by providing systematic
coverage of the vStr with high-density silicon probes, separately
analyzing the low- and high-gamma bands and different behaviors, as
well as providing a causal manipulation known to disrupt piriform
gamma oscillations.

Establishing the source of vStr gamma oscillations is important for at
least two distinct reasons. First, knowledge of the source directly
informs the interpretation of recorded signals. Because vStr neurons
can phase lock to gamma oscillations \cite{Berke2009b, Kalenscher2010,
  VanderMeer2010b,Howe2011, Dejean2016} and vStr ensemble spiking can
predict gamma oscillation frequency \cite{Catanese2016}, the vStr LFP
clearly contains at least some information about local (spiking)
activity. However, if vStr LFPs contain a component volume-conducted
from the piriform cortex, as we have shown, then some changes in the
LFP may reflect processing in piriform cortex, rather than local
processing in vStr. We expand on the issue of how to reconcile LFP
volume conduction with local phase locking below.

The second reason it is important to identify the source(s) of the
vStr LFP relates to how that signal is controlled. Several studies
have linked properties of the vStr LFP to different task components
such as reward approach, receipt, and feedback processing
\cite{Vandermeer2009a,Cohen2009d}, to trait-level variables such as
impulsivity \cite{Donnelly2014}, translationally relevant
interventions such as manipulations of the dopamine system
\cite{Berke2009b,Lemaire2012,Morra2012} and deep brain stimulation
\cite{McCracken2009,Doucette2015}. Our results imply that in order to
change the vStr LFP, either endogenously or using experimental
manipulations, it may paradoxically be more effective to target
piriform cortex rather than the vStr itself.

The result that gamma-band LFP oscillations in vStr are primarily
volume-conducted from a different structure is in line with the
biophysics of LFP generation, which generally require sink/source
pairs of transmembrane currents to be aligned so that their
contributions may sum spatially to generate systematic changes in the
LFP \cite{Nunez2006,Buzsaki2012}. The striatum, as a non-layered
structure with generally radially symmetric dendritic arbors
(\citeNP{Kawaguchi1995, Tepper2004, Tepper2010}), lacks the
organization conducive to spatial summing of currents. Nevertheless,
there have been reports of local heterogeneity in vStr gamma
oscillations. For instance, Kalenscher et al.\ (2010) and Morra et
al.\ (2012) show example recordings for which specific channels show
phases or amplitudes apparently inconsistent with volume
conduction. When we found such examples in our data, however, they
could be attributed to impedance magnitude or angle changes on
isolated electrode sites; the high-density view afforded by Si probe
recordings can disambiguate these cases. A different body of work has
suggested that high-frequency oscillations can be generated locally in
the vStr because they are affected by infusions of MK801 and lidocaine
into the vStr \cite{Hunt2009a,Olszewski2013}; however, given its
anatomical proximity, it is conceivable that some of the drugs spread
to act on the piriform cortex.

So, what is the correct interpretation of ventral striatal gamma
oscillations in the LFP, given the apparent paradox of evidence for
volume conduction on the one hand (as presented here, stylized in
Figure \ref{fig:schematic}A) and evidence for local phase locking and
ensemble coding to gamma-band LFP on the other? We found no phase
reversals that would indicate local generation of gamma oscillations
in the LFP (Figure \ref{fig:schematic}B), as could be supported by
cortical pyramidal-interneuron circuits as demonstrated in neocortical
areas \cite{Cardin2009,Sohal2009,Siegle2014}, or by cell-intrinsic
resonance \cite{Taverna2007}. A different possibility is that LFP
oscillations result from local transmembrane currents, but are
inherited from inputs to the vStr through the synaptic currents they
generate (Figure \ref{fig:schematic}A). As with the local generation
scenario, our results seem to rule out this possibility, given that
the distribution of gamma power across the vStr does not appear to
match known anatomical distributions (Figure \ref{fig:schematic}D;
reviewed in \citeNP{Groenewegen1999, Humphries2010}), and disappears
with piriform inactivation. The vStr does receive inputs from piriform
cortex, however \cite{Brog1993,Schwabe2004}, which could account for
phase-locking in the vStr: to the extent that piriform cortex inputs
are effective in driving vStr spiking, then that spiking would be
expected to lock to the field potential originating in the same source
structure (Figure \ref{fig:schematic}C).

\begin{figure}[h]
\centering
\includegraphics[width=\textwidth]{Naris_Figure_7_Feb}
\caption{Four possible scenarios for the generation of gamma-band
  oscillations in the vStr. \bsf{A}: Volume conduction originating in
  the adjacent piriform cortex would show no clear phase reversals
  within the vStr. \bsf{B}: Local mechanisms such as cell-intrinsic
  currents and circuitry produce local gamma oscillations within the
  vStr circuit.  \bsf{C}: Local sources matching the anatomical
  heterogeneity of the vStr, here idealized with two different
  afferent sources (orange and magenta). \bsf{D}: Rhythmic inputs from
  the adjacent piriform cortex lead to local generation within the
  vStr, which should follow anatomical projection densities.  Our data
  did not show phase reversals within the vStr ruling out local
  generation by cell-intrinsic or multiple synaptic inputs.
  Inactivation of the piriform cortex greatly reduced gamma
  oscillations across the vStr making volume conduction the most
  plausible source of vStr gamma oscillations in the LFP, but does not
  rule out inherited inputs from the piriform, though a lack of phase
  reversals makes this unlikely.}
\label{fig:schematic}
\end{figure}

The above interpretation of vStr gamma oscillations has implications
for a number of avenues of research involving the vStr. For instance,
a long-standing notion is that vStr may provide a ``switchboard''
between inputs from prefrontal cortex, amygdala, and hippocampus
\cite{Odonnell1995,Gruber2009}; LFP oscillations are a major candidate
for implementing and/or reflecting such functions \cite{Fries2005a,
  Fries2015}. Our results suggest that vStr circuits are unlikely to
contain the ``controls'' that determine the timing and frequency of
vStr gamma oscillations. Instead, gamma oscillations volume-conducted
and/or inherited from a common piriform source may be a powerful
synchronizing drive of neural activity in the rodent limbic system. In
particular, LFP synchrony in the gamma band across limbic structures
such as prefrontal cortex, orbitofrontal cortex, ventral hippocampus,
and amygdala \cite{VanWingerden2014,Harris2015,Catanese2016}, may, at
least in rodents, be shaped by piriform input. Given the much larger
distance from the human vStr to piriform cortex, and the widespread
use of relatively local referencing in depth electrode recordings, it
seems a priori unlikely that gamma oscillations in the human vStr are
volume conducted from piriform cortex.  More generally, however, there
is at least some evidence that lateralized nasal breathing affects
both the EEG signal and various aspects of cognitive performance
\cite{Block1989,Zelano2016}; intracranial EEG recordings in epilepsy
patients show a connection between nasal breathing and increases in
power of human delta (0.5-4 Hz), theta (4-8 Hz), and beta (13-34 Hz)
oscillations in the piriform, amygdala and hippocampus. Although gamma
activity has been linked to respiration in the olfactory circuit in
rodents \cite{Gault1963}, the time course of vStr gamma power rules
out respiration as the only factor controlling gamma oscillations in
the vStr LFP. For instance, several studies have noted strong
suppression of gamma power as animals run, compared to rest
\cite{Vandermeer2009a,Malhotra2015a}.

Our study has a number of limitations: we chose to focus on gamma-band
oscillations for several reasons, including the high consistency with
which these oscillations can be probed across multiple species
(rodents and humans in particular), because it is the vStr oscillation
band which has received the most attention in terms of behavioral
correlates and relationship to spiking activity, and because gamma
oscillations are plentiful during rest and well as during
behavior. However, clearly it would be of interest to determine the
sources of other oscillations in the vStr LFP, such as delta, theta
and beta, which have all been linked to local spiking activity and
behavior \citep{VanderMeer2011,Howe2011,Stenner2015,
  Malhotra2015a}. The data we recorded as part of this study did not
reliably contain clearly identifiable epochs with these oscillations,
so this is an avenue for further work. Also, our naris occlusion
procedure likely affects olfactory areas in addition to piriform
cortex, such as the olfactory bulb and the olfactory tubercle (in rats: 
\citeNP{Zibrowski1997}); however, owing to its large size, convoluted
shape and positioning at the ventral surface of the brain, piriform
cortex is difficult to target with higher specificity. Centrifugal
afferents from the entorhinal and piriform cortices and olfactory
tubercle are capable of modulating olfactory bulb gamma even in the
absence of the main peduncle input \citep{Gray1988}, yet we see a
strong suppression of gamma power, suggesting that the entire circuit
is sufficiently impaired.  Despite this limitation, we point to the
convergence between the naris occlusion experiment and the power
gradient observed in the probe recordings to support the most
parsimonious interpretation that gamma LFP oscillations in the vStr
originate in piriform cortex.

In closing, we wish to stress an important point: the above conclusion
that vStr gamma LFP oscillations are volume-conducted from piriform
cortex does {\it not} mean oscillations in the vStr LFP are not
important or an epiphenomenon. As pointed out earlier, the spiking of
vStr neurons shows clear oscillatory signatures, including
intrinsically generated resonance in the gamma range
\cite{Taverna2007}. Indeed, given that pretty much any input to the
vStr is known to have oscillatory activity at the LFP and spiking
levels, it would be hard to imagine how vStr activity would not itself
also show oscillations, which in turn can be used as an access point
to define and manipulate specific functional sub-populations and state
changes, as has been tremendously successful in other areas (\citeNP{Pesaran2002, Colgin2009a, Bosman2012}). Our results should motivate care in the
interpretation of the vStr LFP, and suggest future work in
determining how olfactory inputs may shape activity not just in the
vStr but other limbic structures.

\bibliography{jec_etal_naris}
\end{document}
